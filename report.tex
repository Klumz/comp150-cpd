% Please do not change the document class
\documentclass{scrartcl}

% Please do not change these packages
\usepackage[hidelinks]{hyperref}
\usepackage[none]{hyphenat}
\usepackage{setspace}
\doublespace

% You may add additional packages here
\usepackage{amsmath}

% Please include a clear, concise, and descriptive title
\title{First semester - Progress, challenges, actions}

% Please do not change the subtitle
\subtitle{COMP150 - CPD Report}

% Please put your student number in the author field
\author{1607934}

\begin{document}

\maketitle

\section{Introduction}

Currently, it is uncertain which career path/goal I intend to choose, however I am confident of it being in the games industry. I am interested in continuously developing my programming knowledge, as well as learning the Agile ways and the games industry in general. Challenges I have encountered are as follow: understanding code structure, group work engagement, independent work research, using Git systems such as branching and the use of version control, and presentation skills. 

\section{Understanding code structure}
One of the major things that has been both apparent and challenging for me is understanding code structure. This is obviously a vital skill to have, as it covers everything to do with computing. I am able to understand simple code processes like the ones we did during the tinkering-graphics lectures, although anything further than that has been completely incomprehensible to me. Nothing sticks, even when my tutors and colleagues explained everything in great detail. As expected, this overwhelmingly affected my performance and overall motivation when I worked in teams where I didn't have much to offer at all, forcing my colleague(s) to do most of the work, which I found awfully discouraging and disappointing. Even more so since I see myself as a committed person. To address this, I have decided to practice more during the break in order to catch up, especially since we will be working with the BA's next year. Thereafter, I'd like to compare the results with the current lines of code I have, which should indicate my progress of learning to understand code structure or programming in general.  

\section{Second Key Skill}

Write about 200 words. As above.

\section{Third Key Skill}

Write about 200 words. As above.

\section{Fourth Key Skill}

Write about 200 words. As above.

\section{Fifth Key Skill}

Write about 200 words. As above.

\section{Conclusion}

Write your conclusion here. Though the conclusion should be brief, no more than 100 words, it should do more than merely summarise the report. Focus on the five SMART actions that you intend to take in order to overcome any challenges and/or obstacles. Contextualise how this will help you towards your intended career goal and how this may improve your project for the next semester.

\bibliographystyle{ieeetran}
\bibliography{references}

\end{document}
