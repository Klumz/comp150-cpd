% Please do not change the document class
\documentclass{scrartcl}

% Please do not change these packages
\usepackage[hidelinks]{hyperref}
\usepackage[none]{hyphenat}
\usepackage{setspace}
\doublespace

% You may add additional packages here
\usepackage{amsmath}

% Please include a clear, concise, and descriptive title
\title{Second semester -  Personal Development}

% Please do not change the subtitle
\subtitle{CPD Report - 2nd semester}

% Please put your student number in the author field
\author{1607934}

\begin{document}

\maketitle

\section{Introduction}


\section{Using Unreal Engine}
We have spent a considerable amount of time in Unreal Engine 4. In fact, it wouldn't be far off to say that we've had more incentive to use and learn the engine in our projects than spend time in the more `standardised' lectures, as we've been provided with various tasks that require its use. I personally have grown quite fond of it, as it is something I can see myself go in depth with both for our course and personal purposes. I also really enjoy the whole blueprint-aspect to it, despite how messy it gets. I even believe it to be a good way of learning traditional programming as it visually shows you the logic and functionality, something that can be eventually transferred over to the more coding mindset. With that said, BECAUSE of how easy it is, I find myself leaning more towards it resulting in neglection of traditional coding -- something I want to avoid. To address this, I have decided to spend more time on the coding side of things with aid from Pluralsight's tutorials and the UE4 C++ coding standards and learn how to incorporate it in with blueprints to understand both sides. 

\section{Costum controller}
Another task worth mentioning is our costum controller for our comp140 game. Researching the different kinds of controllers others have created is fun, even more so when they have unusual designs. However, due to this very reason, it has become harder to come up with an original concept - or at least one that stands out enough. Additionally, it would have to be realistic and would have to fit within my skill level. I emphasise this due to the uncertainty of my controller idea, which I realise might be too ambitious. I will have to talk to my peers about this in hopes of learning how I will actually create it and to decide whether it is realistically achievable or not. 

\section{Academic writing}
A skill I have observed others often loathe, which I completely understand. It's a necessary evil for many. While I personally am interested in improving my academic writing, I often find myself neglecting it. This is due to my realisation that I find more joy in reading other's research work than formulating my own. It involves constant practice and iterations. Especially now that I'm in university level, it would be wise to commit time into this, even more so considering how the games industry contains a numerous amount of controversial topics that need and breed attention, such as work practices (e.g. crunch time), accessibility, and so on. Additionally, this skill touches on my personal mindset where I cannot for the life of me express my feelings or thoughts in a `higher-level' manner (even more so verbally). Currently, I find the way I write to be on a fairly basic level with room for improvement, which is why I will try to improve by continuing to absorb other's research work, structure and eventually apply my own opinions and ideas that, of course, will be backed by evidence which is something I also need to work on. 

\section{Time allocation for personal and group projects}
I kept leaning towards our group project with the BAs when it came to doing work, as opposed to allocating time between the different module projects. It wasn't until long after that I realised I barely started with the other projects due to this, which forced me to postpone the group project temporarily and focus primarily on the other tasks (some of which had close deadlines). As a result, my time allocation skills must be improved to avoid similar situations from happening in the future. To improve upon this, I will consider whether the task (e.g. implementing a game mechanic) is worth fully investing my time in, and compare that with other tasks that also need to be done. If I determine the one that requires less effort than the previous, then I do that one first and move on to the next rather than reluctantly doing it bits by bits with little progress.  

\section{Prioritisation}
This goes 

\section{Conclusion}


\end{document}
